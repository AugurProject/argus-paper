% !TEX program = xelatex
\documentclass[11pt,a4paper]{article}
\usepackage{polyglossia}
\usepackage{fontspec}
\usepackage{subfig}
\usepackage{url}
\usepackage{hyperref}
\usepackage[round]{natbib}

% natbib link joining; somewhat breaks \cite, \citet
\makeatletter
\renewcommand\hyper@natlinkbreak[2]{#1}
\makeatother

\usepackage{geometry}
\geometry{%
	includeheadfoot,
	margin=1in
}

\def\TODO{{\bf ??? }}

\title{Argus: Deciding Questions about Events \\ (working paper)}
\author{Petr Baudiš \\ Ailao}

\begin{document}
\maketitle

\begin{abstract}%
	We consider the problem of reliably answering yes/no questions
	about events based on news sources in settings of the Argus QA
	system and the Syphon question pre-validator.
	We give a brief overview of the relevant state-of-art methods
	in the academic field of Natural Language Processing,
	requirements for building a solid state-of-art system
	and difficulties we may anticipate in the road ahead.
	Based on this, we propose a basic architecture of the Argus
	system, roadmap to achieve baseline performance, and a basic
	version of the Syphon system.
\end{abstract}

\vspace{3ex}

{\itshape

In the working paper you should be sure to highlight the key aspects of uncertainty and anticipated difficulty.
	
And also, of course, the need for +alpha development - to whatever extent you
can detail what specifically is known and unknown that future development would
help clarify. Bringing in +that recent Cornell research might also be helpful
if you consider it valid (I know you were kind of skeptical of it).

Another thing I neglected to mention - I know they're not exactly the same, but it would also be very good to whatever extent you can to
point to YodaQA as a kind of preliminary proof of concept and include any mention you can of the lessons learned from building that and how those
lessons would/might come in to play with building the "yes/no/unclear" Argus option.
}

about us.  yodaqa.

paper outline.

\section{QA System in Machine Learning Concept}

dataset

performance measures

initial targets

\section{Machine Learning for Answering Yes/No Questions}

yes/no questions == text entailment (of question and its negation).

\section{Structure of Our Problem}

concrete system proposal.

question classes.

anticipated difficulties.

\section{How Can We Know What Do We Know?}

syphon outline.

\section{Infrastructure}

news gathering etc.

distributed system prospects.

\bibliographystyle{plainnat}
\bibliography{qa}

\end{document}
